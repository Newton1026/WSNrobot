\section{Experiments}\label{sec:Experiment}
Three experiments were completed

   \begin{figure}
   \centering
\begin{overpic}[width =\columnwidth]{drone_base.pdf}\end{overpic}
\caption{\label{fig:OverviewImage}
The seismic drone's sensor base  consists of four geophones, a Geospace Seismic Recorder (GSR) and a LiPo (14.8V, 0.5Ah, 4 cells) battery.
\emph{ TODO: add a ruler for scale.  Can you make both images the same size?}
}
\end{figure}



\subsection{Seismic Survey Comparison}
%describe Feb 19th tests.  Li Chang has documentation on these

The purpose of this experiment is to compare the proposed system's (Seismic Drone) performance with the currently available systems (traditional cabled systems). The seismic drone lifted off, flew to the same locations as four cabled geophones, and was coupled to the recording equipment to obtain graphs for comparison.

\textbf{Materials Required:} 
\begin{center}
 \begin{tabular}{||c c c||} 
 \hline
 S No. & Materials & No. of Units \\ [0.5ex] 
 \hline\hline
1 &	Sledge Hammer &	1 \\ 
 \hline
2 & Traditional Geophones &	10 \\
 \hline
3 &	GSR	& 1 \\
 \hline
4 & Battery,14.8V,10Ah &	1 \\
 \hline
5 &	Seismic Drone &	1 \\ [1ex] 
 \hline
\end{tabular}
\end{center}

\textbf{Procedure:}
\begin{enumerate}
\item Plant the geophones vertically into the ground, 5m apart from one another. Ensure the coupling with the ground is satisfactory.
\item Attach the series geophone bundle to the GSR (Geospace Seismic Recorder).
\item Attach the GSR to the battery for power.
\item Fly drone and land it approximately next to the 1st stationary geophone.
\item At approximately 10m from the geophone setup use the sledge hammer to strike the ground, the impact causes vibrational waves that propagate below the earth's surface and is detected by the geophones.
\item Repeat steps 4 and 5, landing the drone next to each geophone.
\item Save data files from both the drone and the stationary geophones.  In software line up the hammer strikes. Display data for one test with all the stationary geophones overlayed with the data from each position of the seismic drone.
\end{enumerate}


\textbf{Results}


   \begin{figure}
   \centering
\begin{overpic}[width =\columnwidth]{TraditionalCabeledSystem.pdf}\end{overpic}
\caption{\label{fig:OverviewImage}
A sketch of the traditional geophone system,used extensively for Seismic data acquisition..
}
\end{figure}
 \begin{figure}
   \centering
\begin{overpic}[width =\columnwidth]{SeismicDrone.pdf}\end{overpic}
\caption{\label{fig:OverviewImage}
A sketch of the proposed drone setup which can replace manual laborers during seismic surveys.
}
\end{figure}
   \begin{figure}
   \centering
\begin{overpic}[width =\columnwidth]{OXEND.png}\end{overpic}
\caption{\label{fig:OverviewImage}
2 plots showing comparison with traditional geophone system for (1) hard surface, and (2) dirt surface
}
\end{figure}


%\subsection{Recording seismic disturbance using onboard GSR}
%To ensure all measurements had the same attenuation and were synchronous, the previous test  connected the seismic drone to a cabled system.  This experiment demonstrates that the onboard GSR records seismic disturbances.

   \begin{figure}
   \centering
\begin{overpic}[width =\columnwidth]{OXEND.png}\end{overpic}
\caption{\label{fig:OverviewImage}
Data recorded by the cable-free seismic drone
}
\end{figure}

\subsection{Wireless transmission of seismic disturbances}
%describe the Arduino blue-tooth solution
The purpose of this experiment is to compare the seismic drone system's performance with the \textbf{Seismic Wireless Sensor Network Drone} (SWSN Drone).

\textbf{Materials Required:}
\begin{center}
 \begin{tabular}{||c c c||} 
 \hline
 S No. & Materials & No. of Units \\ [0.5ex] 
 \hline\hline
1 &	Sledge Hammer &	1 \\ 
 \hline
2 & SWSN Drone &	10 \\
 \hline
3 &	Seismic Drone &	1 \\ [1ex] 
 \hline
\end{tabular}
\end{center}

   \begin{figure}
   \centering
\begin{overpic}[width =\columnwidth]{OXEND.png}\end{overpic}
\caption{\label{fig:OverviewImage}
One plot from the  Arduino blue-tooth equipped quad copter.
}
\end{figure}

\textbf{Procedure :-}\\
1.	Fly the seismic drone and land it around the first survey location.\\ 
2.	Use a sledge hammer to strike the ground, the impact causes vibrational waves which propagates below the earth surface and is detected by the geophones.\\
3.	Repeat steps one and two for all survey locations.\\ 
4.	Repeat steps one, two and three for the seismic wireless sensor network drone system.\\
5.	Save the data file obtained from the seismic drone and the seismic wireless sensor network drone system (obtained wirelessly). In software line up the hammer strikes. Display data from both the test overlaid on each other for all survey locations.\\ 

\textbf{Results}



\subsection{Accuracy of autonomous landing with geophone setup}
Seismic exploration depends on accurate placement of geophones over a large geographic area.  This experiment tested the accuracy of autonomous landing of the fully loaded seismic drone system verses the manual landing controlled by a piolot using an RC transmitter. The purpose of this experiment is to compare the accurate placement of geophones using manual control and autonomous control.

\textbf{Materials Required :-}
\begin{center}
 \begin{tabular}{||c c c||} 
 \hline
 S No. & Materials & No. of Units \\ [0.5ex] 
 \hline\hline
1 &	Mobile Phone &	1 \\ 
 \hline
2 & SWSN Drone &	10 \\
 \hline
3 &	Seismic Drone &	1 \\ [1ex] 
 \hline
\end{tabular}
\end{center}


\textbf{Procedure:}
\begin{enumerate}
\item Mark the landing location with an x using a red insulating tape.
\item Try to land the seismic drone manually at the center of the marked x location using the wireless RF transmitter.
\item After landing, measure the displacement from the center of the seismic drone to the center of the x location.
\item Repeat steps 2 and 3 ten times.
\item Obtain the mean and variance using the displacement values.
\item Use the mobile phone (tower app) to land the seismic drone autonomously at the center of the $x$ location.
\item Measure the distance between the center of the seismic drone to the center of the $x$ location.
\item Repeat the above two steps ten times.
\item Obtain the mean and variance using the displacement values.
\item Compare the average mean and variance values for manual and autonomous control.
\end{enumerate}

\textbf{Results}


   \begin{figure}
   \centering
\begin{overpic}[width =\columnwidth]{OXEND.png}\end{overpic}
\caption{\label{fig:AutonomousLandingImage}
The seismic drone was commanded to land at the location marked with a green 'x'.  The actual positions are shown with blue 'o'.  The mean and $\pm$1 standard deviation ellipses are drawn in red.
}
\end{figure}


\subsection{Coupling in various soils}
  
We land the drone in various soils, and measure the penetration using a penetrometer.

\textbf{Materials Required :-}\\
\textbf{Procedure :-}\\
\textbf{Results :-}\\



  \begin{figure}
   \centering
\begin{overpic}[width =\columnwidth]{OXEND.png}\end{overpic}
\caption{\label{fig:OverviewImage}
(left) Photographs of seismic drone geophone feet in different soils, with a ruler visible. (right) plot of penetration depth for three different soil types.
}
\end{figure}
