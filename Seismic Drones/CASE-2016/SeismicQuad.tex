%  compress using: gs -sDEVICE=pdfwrite -dCompatibilityLevel=1.4 -dNOPAUSE -dQUIET -dBATCH      -sOutputFile=foo-compressed.pdf SeismicDrones.pdf
\documentclass[letterpaper, 10 pt, conference]{ieeeconf}
%\IEEEoverridecommandlockouts
%\documentclass[conference]{IEEEtran}
\newcommand{\subparagraph}{}
\usepackage{epsfig,graphicx,cite}
\usepackage{psfrag}
\usepackage[small,compact]{titlesec}
\usepackage{wrapfig}
\usepackage{mathrsfs}
\usepackage{bm}
\usepackage{cite,url,subfigure,epsfig,graphicx}
\usepackage{verbatim,amsfonts,amsmath,amssymb}
\usepackage{fancyhdr}
\usepackage{mathbbold}
\usepackage{bbm}
\usepackage{mathrsfs}
\usepackage{amsfonts}
\usepackage{cite,url,subfigure,epsfig,graphicx}
\usepackage{amssymb,amsmath,bm,makecell}
\usepackage{indentfirst}
\usepackage{overpic}
\newcommand{\figwid}{0.22\columnwidth}

\usepackage{amsmath}
\usepackage{algorithm}
\usepackage[noend]{algpseudocode}

\usepackage[T1]{fontenc}
\usepackage[utf8]{inputenc}
\usepackage{authblk}



\usepackage{mathtools}
\usepackage[font=footnotesize]{caption}
\usepackage{amsmath}
\usepackage{amssymb}
\usepackage{tabulary}
\usepackage{booktabs}
\usepackage{framed}
\usepackage{fancyhdr}
%\usepackage[hypertex]{hyperref}
\usepackage[hidelinks]{hyperref}
%\IEEEoverridecommandlockouts
\usepackage{cite,url,subfigure,epsfig,graphicx}
\usepackage{times,verbatim,amsfonts,amsmath,color}
%\newtheorem{definition}{\textbf{Definition}}
%\newtheorem{lemma}{\textbf{Lemma}}
%\newtheorem{proof}{\textbf{Proof}}
%\newtheorem{theorem}{\textbf{Theorem}}
%\newtheorem{example}{\textbf{Example}}
%\newtheorem{proposition}{\textbf{Proposition}}
%\newtheorem{remark}{\textbf{Remark}}
%\newtheorem{corrolary}{\textbf{Corrolary}}
%\newtheorem{ex}{\textbf{EX}}
\usepackage{overpic}
\graphicspath{{./},{./pictures/}}
\setcounter{secnumdepth}{4}
\setcounter{tocdepth}{4}
\usepackage[table,xcdraw]{xcolor}
\newcommand{\todo}[1]{\vspace{5 mm}\par \noindent \framebox{\begin{minipage}[c]{0.98 \columnwidth} \ttfamily\flushleft \textcolor{red}{#1}\end{minipage}}\vspace{5 mm}\par}
\let\labelindent\relax \usepackage{enumitem}

\begin{document}
%
% paper title
% can use linebreaks \\ within to get better formatting as desired
\title{Seismic Surveying with Drone-Mounted Geophones } 

\author[1]{\rm Srikanth K. V. Sudarshan}
\author[1]{\rm Li Huang}
\author[2]{\rm Li Chang}
\author[2]{\rm Robert Stewart}
\author[1]{\rm Aaron T. Becker}
\affil[1]{Department of Electrical and Computer Engineering}
\affil[2]{Department of Earth and Atmospheric Sciences}
\affil[ ]{University of Houston}
\affil[ ]{4800 Calhoun Rd, Houston, TX 77004}
\affil[ ]{\textit {\{skvenkatasudarshan, lhuang21, lchang13, rrstewar, atbecker\}@uh.edu}}
% make the title area
\maketitle

%(http://www.eoearth.org/view/article/155968/) Seismic exploration is the search for commercially economic subsurface deposits of crude oil, natural gas, and minerals by the recording, processing, and interpretation of artificially induced shock waves in the earth
\begin{abstract}

Seismic imaging is the primary technique (and industries) for subsurface exploration. It involves generating a vibration which propagates into the ground, echoes, and is recorded using motion sensors. Often sites of resource or rescue interest may be difficult or hazardous to access. In addition, traditional seismic imaging techniques rely heavily on manual labor to plant sensors, lay miles of cabling, and then collect the sensors. Thus, there is a substantial need for unmanned sensors that can be deployed by air and potentially in large numbers. This paper presents a working prototype of autonomous drones equipped with geophone vibration sensors that can fly to a site, land, listen for echoes and vibrations, store the information on-board, and subsequently return to home base.
The design uses four geophone sensors (with spikes) in place of the landing gear.  This provides a stable landing attitude, redundancy in sensing, and ensures the geophones are oriented perpendicular to the ground. The paper describes hardware experiments demonstrating the efficacy of this technique and a comparison with traditional manual techniques. The performance of the seismic drone was comparable to a well planted geophone, proving the drone mount system is a feasible alternative to traditional seismic sensors.

\end{abstract}
%\begin{IEEEkeywords} Data Acquisition, Geophones, Quadcopters, Seismic Exploration \end{IEEEkeywords}

\section{Introduction}\label{sec:introduction}

\begin{figure}
\centering
\begin{overpic}[width =\columnwidth]{introduction_fig_1.pdf}\end{overpic}
\caption{\label{fig:introimg}
 Comparing manual and robotic geophone placement. a.) Currently, geophones are planted manually. A well planted geophone is aligned with the gravity vector. b.) Traditional methods require extensive cables to connect geophones to the seismic recorders and batteries. c.) The \emph{Seismic Drone} in this paper is an autonomous unit requiring no external cables. This paper presents an automated  process for sensor deployment and retrieval. \href{https://youtu.be/yxdUEX0SPyw}{See video of prototype at~\cite{SDV16}}.
}
\end{figure}
 
Hydrocarbons (coal, oil, natural gas) 
supplied more that 66\% of the total energy consumed according to an estimate by IEA (International Energy Agency)~\cite{IEA16}.
 Millions of dollars are invested in seismic exploration. Avoiding hazards and maintaining safety during exploration is necessary because hydrocarbons are inflammable.
Traditional exploration involves planting geophones (sensors)
into the soil and detecting seismic disturbances caused
from vibrating trucks or dynamite detonations which act as a source of vibration. 
As these vibrations propagate they are reflected and refracted by different layers below the surface. Geophones sense these vibrations and store the data on board or send it to a data processing unit called the \emph{StrataVisor}. The
data obtained describes the amplitude of the plastic waves
generated by the source over a period of time.  Instead of randomly searching for hydrocarbons, explorations are carried out using elaborate technical procedures, equipment, and skilled labor over a large area. This increases the possibility of discovering hydrocarbon-reserves in an optimal fashion, using the data obtained. 
Cables are used to connect the seismic recorder and the sensors, but cabling cost is proportional to area, and certain terrains are inaccessible, such as jungles or wetlands. The exploration process involves repeated manual deployment and redeployment of sensors. Applying current advancements in robotics and automation we would be able to reduce the cost, decrease time and increase precision in sensing seismic waves. Fig.~\ref{fig:introimg} displays the major drawbacks of traditional seismic exploration and the solution presented in this paper, a  flying UAV for geophone placement and recovery.

Drones or unmanned aerial vehicles (UAVs) are flying
platforms with propulsion, positioning, and independent self control.
As drone technology improves and regulations are
adopted, there are major opportunities for their use in scientific measurement, engineering studies, education and agriculture ~\cite{tripicchio2015towards}. In particular,
measuring mechanical vibrations is a key component of many
fields, including earthquake monitoring, geotechnical engineering,
and seismic surveying. Seismic imaging is one of the
major techniques for subsurface exploration
and involves generating a vibration which propagates
into the ground, echoes, and is then recorded using motion
sensors. There are numerous sites of resource or rescue interest
that may be difficult or hazardous to access. In addition, the abundance of survey sites require a great deal
of hand labor. Thus, there is a substantial need for unmanned
sensors that can be deployed by air and potentially in large
numbers. 
%This paper presents working prototypes of an seismic drone that can fly to a site, land, then listen for echoes and vibrations, transmit the information, and subsequently return to its home base.
The goal of this paper is to design, build, and demonstrate
the use of motion sensing drones for seismic surveys, earthquake monitoring, and remote material testing. 

Section~\ref{sec:RelatedWork}  gives an overview of  the current state-of-the-art technology available in the industry and why it is useful to complement current technology  with the Seismic Drone.
Section~\ref{sec:Experiment} describes the hardware experiments and results performed, validating that the seismic drone is a reliable option. Section~\ref{sec:Conclusion}, concludes with future work.


%
\section{Overview and Related Work}\label{sec:RelatedWork}

This paper presents a \emph{heterogeneous sensor system} for automatic sensor deployment. The goal is to overcome the drawbacks of deploying seismic sensors manually. 
In previous work \cite{sudarshan2015using}, we demonstrated a UAV equipped with four geophone sensors as landing gear.
This UAV automated sensor deployment by flying to GPS waypoints to obtain seismic data. 
This solution had several limitations.
Magnet-coil geophones contain a permanent magnet on a spring inside a coil. Voltage across the coil is proportional to velocity.  Beneath the coil housing is a metal spike.  Geophones are \emph{planted} by pushing this metal spike into the group, which improves coupling with the ground to increase sensitivity. The magnet-coil must be aligned with the gravity vector. Mialignment reduces the signal proportional to the cosine of the error.

The geophones in  \cite{sudarshan2015using} were connected to the UAV, causing four problems
(1) one UAV was required for each additional sensor,
(2)  the force for planting the geophone was limited by the weight of the UAV,
(3) the platform required a level landing site,
(4) the magnets in the geophones distort compass readings, causing landing inaccuracy when autonomous.

The proposed heterogeneous sensor system separates the sensing units from the UAV.
This reduces the cost per sensor. 
Dropping the geophones enables increasing geophone penetration by increasing drop height and eliminates the necessity for a level landing site.
The new design also increases separation between geophones and the UAV.

\subsection{Overview of Seismic Sensing Theory}

\begin{figure}
\centering
\begin{overpic}[width=\columnwidth]{overview.pdf}\end{overpic}
\caption{\label{fig:sensor_types}
 Comparing state-of-the-art seismic survey sensors a.) Traditional cabled system, the geophones (sensors) are connected in series to the seismic recorder and battery. b.) Autonomous nodal systems, each geophone has a seismic recorder and a battery making each geophone ``autonomous" from the other geophones.}
 \vspace{-2em} 
\end{figure}



During seismic surveys a source generates seismic waves that propagate under the earth's surface. 
These waves are sensed by geophone sensors and are recorded for later analysis to detect the presence of hydrocarbons. 
Fig.~\ref{fig:sensor_types} illustrates the components of current sensors. 

For clarity, the following section discusses 1D waves.  The full 3D equations can be found in many geophysics textbooks, for example~\cite{shearer2009introduction}.
Geophone sensors sense the vertical external displacement $U$ caused by vibrational waves that propagate with a velocity $c$ in the positive and negative $x$-directions. 
Typical seismic wave velocities are in the range $2-8$ km/s.
These waves are represented by the 1D differential equation
\begin{equation}
\frac{\partial^{2}{U}}{\partial^{2}{t}} = {c}^{2}\frac{\partial^{2}{U}}{\partial^{2}{x}}.
\end{equation}

Its general solution is given by
\begin{equation}
U(x,t) = f(x \pm c t).
\end{equation}
The equations stated above are a generalized representation of a vibrational wave. For example, a vibrating string would satisfy the equation. 
\begin{equation}
{c}^{2} = F/\rho,
\end{equation}
here $F$ is the vibration force and $\rho$ is density.
This hyperbolic equation is challenging to solve because sharp features can reflect off boundaries.
\todo{Why do we have any of the previous equations?  They would only be useful if we showed an equation on how they are inverted...}

% This is a $3$-D seismic wave equation that scales in complexity and connects the motion of the moving coil with the relative magnetic flux, for a displacement caused by an external source.
%\begin{equation}
%m\frac{\partial^{2}{\xi}}{\partial^{2}{t}}+c\frac{\partial{\xi}}{\partial{t}}+k\xi = m\frac{\partial^{2}{U}}{\partial^{2}{x}}-Bli
%\end{equation} 
%Here $\xi$ is the coil displacement, $k$ is the spring constant, $m$ is the moving mass of the coil, $c$ is the friction coefficient, $B$ is the magnetic flux density, $l$ is the length of coil wire, $i$ is the current. 

\subsubsection{Cabled Systems}
 Traditional \emph{cabled systems} are extensively used for seismic data acquisition in hydrocarbon exploration. A group of sensors (geophones) are connected to each other in series using long cables, and this setup is connected to a seismic recorder and a battery. The seismic recorder consists of a micro-controller which synchronizes the data acquired with a GPS signal and store the data on-board. Generally, four-cell Lithium Polymer (LiPo, 14.8V, 10Ahrs) batteries are used to power this system. This method of data acquisition requires many manual laborers and a substantial expenditure for transporting the cables. 
 Rugged terrain makes carrying and placing cable labor intensive and the local manual labor pool may be unskilled or expensive.
   
 \subsubsection{Autonomous Nodal Systems}
 Currently, \emph{autonomous nodal systems}~\cite{wood1998distributed} are extensively used for conducting seismic data acquisition surveys in USA. Unlike traditional cabled systems, autonomous nodal systems are not connected using cables. The sensor, seismic recorder, and battery are all combined into a single package called a node, that can autonomously record data as shown in Fig.~\ref{fig:sensor_types}. Even in these systems the data is stored in the on-board memory and can only be acquired after the survey is completed. This is disadvantageous since errors cannot be detected and rectified while conducting the survey. Recently, wireless autonomous nodes have been developed. These systems can transmit data wirelessly as a radio frequency in real time~\cite{jiang2015geophysical}. Yet these systems still require manual laborers for planting the autonomous nodes at specific locations and deploying the large antennas necessary for wireless communication.
 
\subsection{Related Work}

Seismic surveying is a large industry.
The concept of using robots to place seismic sensors dates to the 1980s, when mobile robots placed seismic sensors on the moon~\cite{LSisMSE81}. Postel et al. proposed a mobile robot for terrestrial geophone placement~\cite{DSSMaA14}. Plans are underway for a swarm of seismic sensors for Mars exploration~\cite{MAPL2006}.
Additionally, \cite{muyzert2015marine} and \cite{postel2014drone} proposed marine robots for geophone deployment underwater. 
Other work  focuses on data collection, using a UAV to wirelessly collect data from multiple sensors~\cite{wilcox2013seismic}. 
% Their idea is to have a propulsion module that either be a wheels, tracks, turbines, helicopter blades etc. 
% This robot is used to deploy sensors at specified locations. 
% The system consists of modules for ensuring the sensor is well planted and placed perpendicular to the ground. The modular approach to a deployment system is innovative but might not be practically feasible. 
% While servicing a large area ability to deploy multiple sensors in limited time is a key factor. 
% The approach seems to be well suited for short surveys. 
Our system consists of a multi-agent system approach designed to quickly and efficiently perform a survey.

%Some works relate to surveying an earthquake prone region to collect data as in \cite{dominici2012micro}. 
%The goals of these papers is to perform a survey rather than sensor deployment.
% An interesting proposal was to perform a seismic survey using augmented reality \cite{jones2016seismic}.

\subsection{Sensor networks}
\subsection{Multi-Robot Assignment}
%
\section{Experiments}\label{sec:Experiment}

\begin{figure*}
\centering
\renewcommand{\figwid}{0.4\columnwidth}
\begin{overpic}[width =\figwid]{round_platform_compare.pdf}
\end{overpic}
\begin{overpic}[width =\figwid]{wooden_platform_compare.pdf}
\end{overpic}
\begin{overpic}[width =\figwid]{well_planted_compare.pdf}
\end{overpic}
\begin{overpic}[width =\figwid]{satisfactorily_planted_compare.pdf}
\end{overpic}
\begin{overpic}[width =\figwid]{drone_platform_compare.pdf}
\end{overpic}

\caption{\label{fig:exp_1_pics} Different geophone configurations and setups compared with the seismic drone for analyzing the seismic wave output obtained after triggering the source:
a.) round platform b.) wooden platform c.) well planted geophone d.) satisfactorily planted geophone e.) drone system with sensor platform (Seismic Drone).}
\end{figure*}

The sensor platform of the seismic drone contains four geophones as shown in Fig.~\ref{Sensor_Base} and the current seismic drone can only plant (submerge the spikes) in soft soil. Traditionally, planting the geophones is essential to obtain reliable coupling between the ground and sensor. To overcome the issue of satisfactory coupling we use four geophones that are connected in series. The geophones are placed $20-30$ cm apart, but due to the fast propagation of seismic waves can be considered as four geophones being placed at the same location. Hence instead of one \emph{well-planted} geophone at a particular location, we use four \emph{satisfactorily-planted} geophones  to obtain comparable results. In particular, the alignment platform ensures sensors are perpendicular to the ground. \href{https://youtu.be/yxdUEX0SPyw}{A video of the prototype performing the experiments is available at~\cite{SDV16}}.

We conducted three experiments to prove the seismic drone is a feasible option that can replace current state-of-the-art technology in the field of seismic exploration. The first experiment compares the sensed seismic vibrational wave output from traditional geophones with the seismic drone. This comparison validates the capability  of the proposed system to replace a conventional setup. The second experiment analyzes autonomous flying with and without the sensor platform, to explore the reliability of autonomous flight and the effects of the sensor platform on the command execution capabilities due to signal interference. The third experiment compares soil penetration and the angle of incidence in three different soil types. This is important to ensure quality data despite soil variations and shows that the platform can takeoff, even when the geophones are well planted in soil. Traditional geophone placement requires pushing the geophone spike into the earth to ensure ground-sensor coupling. The quality of a placement is determined by this coupling and the alignment of the spike with gravity vector. Sensitivity decreases with the cosine of the angle from the spike to the gravity vector.

\subsection{Seismic Survey Comparison}

The primary experiment presented in this paper compares the proposed \emph{Seismic Drone} performance with a traditional cabled sensing system. We compare the seismic drone with different variations to understand its performance. The comparison is done with a \emph{well-planted} geophone: a completely planted geophone where the spike is completely beneath the surface \emph{satisfactorily-planted} geophone: spike is partially into the ground. The drone is also compared to a geophone mounted on a \emph{round-platform} made of fiber glass and finally to a geophone mounted on a long rectangular \emph{wooden-platform}. The described setups are shown in Fig.~\ref{fig:exp_1_pics}. Ideally geophones are always well planted into to the ground, the platform setups and satisfactorily planted geophones are tested to test how performance varies with coupling to the ground. Seismic exploration must detect the oscillating seismic wave and sensing quality is a function of coupling. 

A sledge hammer was used as a source to create the seismic vibrations that propagate beneath the surface. The seismic drone was flown to its respective survey location next to the well planted, satisfactorily planted, round platform and wooden platform geophones. The geophones from the traditional, and seismic drone setup were connected to the \emph{Strata-Visor}, a special computer designed for plotting the data acquired from exploration. The sledge hammer was used to strike a vibrating plate attached to the ground thereby creating seismic waves for analysis.

From the results obtained we observe that the amplitude peaks of the seismic drone is similar to the setups (\emph{well-planted, satisfactorily-planted, round-platform, wooden-platform}) as shown in Fig.~\ref{exp_compare}. We observe oscillations in the \emph{round platform} and \emph{wooden platform} since these are not fixed to the surface. Instead of detecting the strike, the platform starts oscillating due to the strike and these oscillations eventually dampen out over time. The performance of the \emph{round platform} and \emph{wooden platform} are poor in comparison to the \emph{well planted} geophone, which is the standard for this experiment. The seismic drone setup and the well planted geophone display excellent similarities in their response. Both the seismic drone and the well planted geophone setup have minimal oscillations, which is an important feature for seismic exploration, this validates the efficiency in coupling with the surface.  

Experiment 1 compares the performance of the seismic drone with other setups. The experiment described above was a one-to-one based comparison, however seismic explorations use thousands of geophones to conduct a seismic survey. Thus Experiment 1 was extended to compare the performance of a traditional cabled $24$ geophone system connected to a $24$ channel seismic recorder and a battery with an autonomous seismic drone. The geophones were planted vertically into the ground, $1$ m apart from one another.  A schematic of the traditional setup is shown in Fig.~\ref{trad_sketch} and the same experiment was repeated for the seismic drone as shown in Fig.~\ref{seisdrone_sketch}. We used a vibrating truck setup to generate the seismic wave. The geophones are well planted, the drone was flown from $1-24$ locations and the readings were taken by generating seismic waves each time. The metal plate was struck $24$ times, once for each location.

\begin{figure}
\centering
\begin{overpic}[width =\columnwidth]{ezp_1_overview.pdf}\end{overpic}
\caption{\label{ezp1_overview}
A survey comparison was performed to obtain the shot gather plots of the traditional cabled system and seismic drone. a.) Overview of the experiment. b.) The vibrating setup strikes the metal plate below and generates vibrational waves. c.) Strata-Visor is a device used to store and process the signals from the cabled system and the seismic drone. d.) The drone system and the cabled system are listening to the vibrational waves and sending their corresponding readings to the Strata-Visor. 
}
\end{figure}
Fig.~\ref{ezp1_overview} describes the important components of the field experiment performed. Results of the seismic survey field test comparison between a $24$ channel traditional cabled geophone system and the seismic drone shown in Fig.~\ref{shot_gather_compare}.  Both the plots were obtained using a \emph{Strata Visor}, a device that can obtain, store and plot the sensed data. It is extensively used with traditional geophone setups because the geophones can only sense vibrational waves and is dependent on other devices for storage and data processing. To allow a fair comparison, the autonomous setup that \emph{can} store the sensed data present on the seismic drone was not used in this experiment. We observe excellent similarity, thereby proving the seismic drone system can compete with state-of-the-art technology in seismic exploration.
 
 %Attach the series geophone bundle to the GSR (Geospace Seismic Recorder).Attach the GSR to the battery for power.Fly drone and land it approximately next to the 1st stationary geophone.At approximately 10m from the geophone setup use the sledge hammer to strike the ground, the impact causes vibrational waves that propagate below the earth's surface and is detected by the geophones.Repeat steps 4 and 5, landing the drone next to each geophone.Save data files from both the drone and the stationary geophones.  In software line up the hammer strikes. Display data for one test with all the stationary geophones overlayed with the data from each position of the seismic drone.

\begin{figure}
\centering
\begin{overpic}[width =\columnwidth]{TraditionalCabeledSystem.pdf}\end{overpic}
\caption{\label{trad_sketch}
A schematic of a traditional 24 geophone system, used extensively for seismic data acquisition.
}
\end{figure}
 \begin{figure}
   \centering
\begin{overpic}[width =\columnwidth]{SeismicDrone.pdf}\end{overpic}
\caption{\label{seisdrone_sketch}
A schematic of a proposed drone setup which can replace manual laborers during seismic surveys.
}
\end{figure}
%\begin{figure}
%\centering
%\begin{overpic}[width =\columnwidth]{DataComparison.pdf}\end{overpic}
%\caption{\label{}
%2 plots showing comparison with traditional geophone system for (1) hard surface, and (2) dirt surface
%}
%\end{figure}

\begin{figure}
\centering
\begin{overpic}[width =\columnwidth]{exp11_compare.pdf}\end{overpic}
\caption{\label{exp_compare} Displays the seismic wave generated by different geophone setups and the seismic drone. a.) Compares the drone setup with different platforms (round and wooden platforms) oscillations in these platforms are not damped quickly since they are not fixed to the ground, The max amplitude values are similar and appear almost simultaneously indicating these setups were placed very close to each other, so no time shift is observed. b.) Compares the drone setup with planted geophones (well planted and satisfactorily planted), we observe mild oscillations in the drone setup compared to the fixed ones since they are planted into the ground. The max amplitude values are similar but do not appear simultaneously, indicating these setups were placed approximately half meter apart, and  hence time a shift occurred.}
\end{figure}



%\subsection{Recording seismic disturbance using onboard GSR}
%To ensure all measurements had the same attenuation and were synchronous, the previous test  connected the seismic drone to a cabled system.  This experiment demonstrates that the onboard GSR records seismic disturbances.

   \begin{figure}
   \centering
\begin{overpic}[width =\columnwidth]{shot_gather_compare.pdf}\end{overpic}
\caption{\label{shot_gather_compare} A \emph{shot gather} plot comparison, the $x$-axis is time in milliseconds and the $y$-axis is the channel number on the Strata-Visor, to which both the setups are connected. Each survey location is 1 m apart and the wave generated from the source propagates beneath the surface. The waves are time shifted from the first channel to the end. a.) shot gather for a traditional cabled system. b.) shot gather for the seismic drone system.
}
\end{figure}


\subsection{Accuracy of autonomous landing with geophone setup}
Seismic exploration depends on accurate placement of geophones over a large geographic area.  This experiment tested the \emph{accuracy} of autonomous landing of the fully loaded seismic drone system compared to the autonomous landing of the drone system without the sensor base.

The drone system used is a 3DR Solo.The seismic drone was commanded to land at the goal location marked with a `$x$' using a blue insulation tape, with and without the sensor platform. The test was repeated ten times to test the accuracy of autonomous landing. The drone uses GPS for landing which is not highly accurate and hence lands at locations close to the goal location. The origin of the coordinate system was marked with a `$x$' using a yellow insulation tape. Measuring tapes were used to measure the location at which the drone landed to the origin.
The sensor base attached to the drone for seismic sensing has four geophones. A geophone has a strong magnet attached to a spring to measure vibrations. These magnets on the sensor base influence the internal compass of the drone system with their strong magnetic fields. This effect can be observed in the plots shown in Fig.~\ref{fig:AutoLandPlots}. The ${1}^{st}$ and ${2}^{nd}$ standard deviation ellipses are much smaller for the drone system without the sensor base than the system with the sensor base. The GPS used by the drone has an accuracy of five meters and we observe the landing locations approximately to be normally distributed. $95$\% of the landings were within $1$ m for the drone system without the sensor base and around $2$ m for the drone system with sensor base. The current landing accuracy is sufficient for seismic exploration. A $2$ m error in distance from the landing site corresponds to an increase or decrease in travel time for the seismic wave by $25$ ms approximately. 

The autonomous planning of the drone is done using a mobile application called \emph{Tower}. This app can be used to plan complex autonomous trajectories, and the drone can perform different tasks at different waypoint locations. 
 
% Try to land the seismic drone manually at the center of the marked x location using the wireless RF transmitter.After landing, measure the displacement from the center of the seismic drone to the center of the x location.Repeat steps 2 and 3 ten times.Obtain the mean and variance using the displacement values.Use the mobile phone (tower app) to land the seismic drone autonomously at the center of the $x$ location.Measure the distance between the center of the seismic drone to the center of the $x$ location.Repeat the above two steps ten times.Obtain the mean and variance using the displacement values.Compare the average mean and variance values for manual and autonomous control.
 
\begin{figure}
\centering
\begin{overpic}[width =\columnwidth]{exp_2_ellipse.pdf}\end{overpic}
\caption{\label{fig:AutoLandPlots}
The plots describe autonomous landing with and without the sensor platform. For 10 landings the landing locations are closer to the goal location for without the sensor platform than with the sensor platform. The mean, 1st std. ellipse and 2nd std. ellipse are shown for both cases.
}
\end{figure}


\subsection{Penetration and Angle with the Horizontal}

This experiment tests the soil penetration capabilities of the seismic drone setup in different soil types. Good coupling with soil is important for obtaining quality data, hence the experiment explores the penetration capability of the setup in common soils. We performed the experiment in grass, sand, and dry clay. The penetration was maximum in sand followed by grass, but the drone could not drive the geophone spike into dry clay, as  shown in Fig.~\ref{fig:DepthPlot}. Failing to penetrate through dry clay was inevitable even with a manual plant. The seismic sensors are highly sensitive and could collect data without penetrating through a surface, if place vertical to the surface. Since the design considers vertical placement of geophones, a seismic analysis could be achieved by landing on any flat hard surface like dry clay, terrace of a building or a cement road. This system could replace humans who risk lives to sense earthquakes or perform quality checks on a partially completed bridge.

The final experiment measured the angle of deviation of the geophone from vertical. Ideally the geophones should be perpendicular to the ground. This is necessary to obtain quality data, since the data loses accuracy with the cosine of this angle. A rule of thumb is to have less than ${5}^{\circ}$ error for a geophone. It is important to land on a flat surface with less than ${10}^{\circ}$ deviation from the horizontal, otherwise the drone will not fly due to difficulties it faces in take off while inclined at an angle. These two constraints complement each other. We collected data of the roll and pitch Euler angles to calculate the deviation from the horizontal using the cross-product of rotation vectors ${R}_{x}(Roll) \times {R}_{y}(Pitch)$, as shown in Fig.~\ref{fig:AnglePlot}.

\begin{figure}
\centering
\begin{overpic}[width =\columnwidth]{depth_box_plot.pdf}\end{overpic}
\caption{\label{fig:DepthPlot}
Box and whisker plots comparing the variations in depth of planted geophones attached to the seismic drone 
}
\end{figure} 

\begin{figure}
\centering
\begin{overpic}[width =\columnwidth]{angle_box_plot.pdf}\end{overpic}
\caption{\label{fig:AnglePlot}
Box and whisker plots comparing the variations in angle of deviation from horizontal of the seismic drone 
}
\end{figure} 
  

%
 \section{Conclusion and Future Work}\label{sec:Conclusion}
This paper presented an autonomous technique for geophone placement, recording, and retrieval. The system enables automating a job that currently requires large teams of manual laborers. Three components were introduced, SeismicDarts, a mobile SeismicSpider, and a deployment unit.
Field and laboratory hardware experiments demonstrated the efficacy of the robotic team compared to traditional techniques. 
The SeismicDart's output were comparable to well-planted geophones. 
For hard surfaces where the SeismicDart could not penetrate, an autonomous alternative was presented, the SeismicSpider.  
The SeismicSpider is mobile, can actively adjust its sensors to ensure ground contact and vertical placement, and can be deployed and retrieved by UAVs.

Autonomous deployment was conducted using GPS, proving human involvement could be minimized by adopting the proposed technique. 
Hardware experiments compared the autonomous system to manual planting and ballistic deployment.
Simulation studies show time and cost savings over traditional manual techniques.

Future systems should be weatherized and optimized for cost, robustness, range, and speed. 



\bibliographystyle{IEEEtran}
\bibliography{./bibs/Match}

% that's all folks
\end{document}


