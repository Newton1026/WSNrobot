 \section{Conclusion}\label{sec:Conclusion}
This paper presented an autonomous technique for geophone placement, recording, and retrieval. The system enables automating a job that currently requires large teams of manual laborers.
 The paper described hardware experiments demonstrating the efficacy of the seismic drone compared to traditional techniques. The drone-sensing platform's output was comparable to a well planted geophone, suggesting the feasibility of the proposed system. Autonomous landing was conducted using GPS, thus displaying closed loop control. This proved human involvement could be drastically minimized by adopting the proposed technique. Angle of penetration was compared between different soil types with deviations of around $2$ deg. This proved the benefits of the sensor platform design and reduced errors in sensor data. The system displayed the ability to penetrate soil types like sand and grass and an inability to penetrate hard types like dry clay, yet it could perform sensing and obtain sensory data.
 
Future drone systems could be designed solely for seismic exploration purposes thereby increasing robustness, increasing flight and stationary periods, and could be weatherized.  
A quad-rotor system in general has limitations in flight time and in the future we would like to separate the sensing platform from the deployment unit to drop and pick up sensing units. Designs could be immobile passive sensing units or mobile active units that create and measure a seismic wave. Given a heterogeneous set of sensing units, further optimization could give insight on each type of sensing unit required. 

 %Pre and post signal processing techniques could be adapted to improve the quality of sensing. Data could be transmitted in real time to ease the exploration process and, identify errors and perform corrections. Transmitting high amounts of data is an issue, novel methods to compress, transmit, receive and interpret is an exciting research direction. Creating path planning algorithms for performing deployment retrieval tasks constrained by the availability of resource is an interesting future direction. It may be more beneficial to deploy one or more sensor packages and return the drone to a home base for charging or have a team of drones that divide the task of sensor deployment and retrieval. The precision on the GPS could be improved by upgrading to a RTK (Real Time Kinematic) or a DGPS (Differential GPS) system. Including other sensors like lasers, sonar, or a vision system could improve precision in deployment, can improve robustness to random disturbances and avoid obstacles. There are many opportunities for future work. A mobile app could be created to perform tailor-made tasks focusing on seismic exploration, which could be used by an human/robot operator to plan an exploration strategy. These ideas could be enforced to make the real system accessible and operational by a minimal work force.

