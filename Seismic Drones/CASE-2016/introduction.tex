\section{Introduction}\label{sec:introduction}

\begin{figure}
\centering
\begin{overpic}[width =\columnwidth]{introduction_fig_1.pdf}\end{overpic}
\caption{\label{fig:introimg}
 Comparing manual and robotic geophone placement. a.) Currently, geophones are planted manually. A well planted geophone is aligned with the gravity vector. b.) Traditional methods require extensive cables to connect geophones to the seismic recorders and batteries. c.) The \emph{Seismic Drone} in this paper is an autonomous unit requiring no external cables. This paper presents an automated  process for sensor deployment and retrieval. \href{https://www.youtube.com/user/aabecker5}{See video of prototype at \cite{}}.
}
\end{figure}
 
Hydrocarbons (coal, oil, natural gas) 
supplied more that 66\% of the total energy consumed on earth
during the past accordng to an estimate by IEA (International Energy Agency)~\cite{IEA16}.
 Millions of dollars are invested in exploration. Avoiding hazards and maintaining safety during exploration is necessary because hydrocarbons are highly-inflammable.
Traditional exploration involves planting geophones (sensors)
into the soil and detecting seismic disturbances caused
from vibrating trucks or dynamite detonations which act as a source of
vibration. As these vibrations propagate to the surface they
are detected by the geophones and the data is stored. The
data obtained describes the amplitude of the plastic waves
generated by the source over a period of time. This data is used to reconstruct the $3$-D layers beneath the surface and the presence of hydrocarbons. Instead of randomly searching for hydrocarbons, explorations are carried out using elaborate technical procedures, equipment, and skilled labor over a large area, thereby increasing the possibility of discovering hydrocarbon-reserves in an optimal fashion. 
Cables are used to connect the seismic recorder and the sensors, but cabling cost is proportional to area, and certain terrains are inaccessible, such as jungles or wetlands. The exploration process involves repeated manual deployment and redeployment of sensors. Applying current advancements in automation would reduce the costs, decrease time and increase precision. Fig.~\ref{fig:introimg} displays the major drawbacks of traditional seismic exploration and the solution presented in this paper, a  flying robot for geophone placement and recovery.

Drones or unmanned aerial vehicles (UAVs) are flying
platforms with propulsion, positioning, and independent self control.
As drone technology improves and regulations are
adopted, there are major opportunities for their use in scientific
measurement, engineering studies, and education. In particular,
measuring mechanical vibrations is a key component of many
fields, including earthquake monitoring, geotechnical engineering,
and seismic surveying. Seismic imaging is one of the
major techniques for subsurface exploration
and involves generating a vibration which propagates
into the ground, echoes, and is then recorded using motion
sensors. There are numerous sites of resource or rescue interest
that may be difficult or hazardous to access. In addition, the abundance of survey sites require a great deal
of hand labor. Thus, there is a substantial need for unmanned
sensors that can be deployed by air and potentially in large
numbers. This paper presents working prototypes of an seismic drone that can fly to a site, land, then
listen for echoes and vibrations, transmit the information, and
subsequently return to its home base.
The goal of this paper is to design, build, and demonstrate
the use of motion sensing drones for seismic surveys, earthquake monitoring, and remote material testing. 

Section~\ref{sec:RelatedWork}  gives an overview of  the current state-of-the-art technology available in the industry and why it is useful to complement current technology  with the Seismic Drone.
Section~\ref{sec:Experiment} describes the hardware experiments and results performed, validating that the seismic drone is a reliable option.Section~\ref{sec:Conclusion}, concludes with future work.

