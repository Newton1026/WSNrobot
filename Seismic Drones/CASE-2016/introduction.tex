\section{Introduction}\label{sec:introduction}

\begin{figure}
\centering
\begin{overpic}[width =0.8\columnwidth]{introduction_fig_1.pdf}\end{overpic}

\caption{\label{fig:introimg}
 Comparing manual and robotic geophone placement. a.) Currently, geophones are planted manually. A well planted geophone is aligned with the gravity vector. b.) Traditional methods require extensive cables to connect geophones to the seismic recorders and batteries. c.) The \emph{Seismic Drone} in this paper is an autonomous unit requiring no external cables. This paper presents an automated  process for sensor deployment and retrieval. \href{https://youtu.be/yxdUEX0SPyw}{See video of prototype at~\cite{SDV16}}.
}
\vspace{-2em}
\end{figure}

Hydrocarbons (coal, oil, natural gas) 
supplied more that 66\% of the total energy consumed according to an estimate by IEA (International Energy Agency) in 2014~\cite{IEA16}.
 Millions of dollars are invested in seismic exploration to find underground hydrocarbons. Avoiding hazards and maintaining safety during exploration is necessary because hydrocarbons are inflammable.
Traditional exploration involves planting geophones (sensors)
into the soil and detecting seismic disturbances caused
by vibrating trucks or dynamite detonations which act as a vibration source. 
As these vibrations propagate they are reflected and refracted by different layers below the surface. Geophones sense these vibrations and store the data on-board or send it to a data processing unit. The data obtained describes the amplitude of the seismic waves at the geophone locations. Instead of randomly searching for hydrocarbons, explorations are carried out using elaborate technical procedures, equipment, and skilled labor over a large area. This increases the possibility of discovering hydrocarbon-reserves in an optimal fashion, using the data obtained. 
Cables are used to connect the seismic recorder and the sensors, but cabling cost is proportional to area, and certain terrains are inaccessible, such as jungles or wetlands~\cite{meunier2011seismic}. The exploration process involves repeated manual deployment and redeployment of sensors. Applying current advancements in robotics and automation could reduce the cost, decrease time and increase precision in sensing seismic waves. Fig.~\ref{fig:introimg} displays the major drawbacks of traditional seismic exploration and the solution presented in this paper, a  flying UAV for geophone placement and recovery.

Drones or unmanned aerial vehicles (UAVs) are flying
platforms with propulsion, positioning, and independent self-control.
As drone technology improves and regulations are
adopted, there are major opportunities for their use in scientific measurement, engineering studies, education and agriculture~\cite{tripicchio2015towards}. In particular,
measuring mechanical vibrations is a key component of many
fields, including earthquake monitoring, geo-technical engineering,
and seismic surveying. Seismic imaging is one of the
major techniques for subsurface exploration
and involves generating a vibration which propagates
into the ground, echoes, and is then recorded using motion
sensors. There are numerous sites of resource or rescue interest
that may be difficult or hazardous to access. In addition, the abundance of survey sites require a great deal
of hand labor. Thus, there is a substantial need for unmanned
sensors that can be deployed by air and potentially in large
numbers. 
%This paper presents working prototypes of an seismic drone that can fly to a site, land, then listen for echoes and vibrations, transmit the information, and subsequently return to its home base.
The goal of this paper is to design, build, and demonstrate
the use of motion sensing drones for seismic surveys, earthquake monitoring, and remote material testing. 

Section~\ref{sec:RelatedWork}  gives an overview of  the current state-of-the-art technology available in the industry and why it is useful to complement current technology  with the Seismic Drone.
Section~\ref{sec:Experiment} describes the hardware experiments and results performed, validating that the seismic drone is a reliable option. Section~\ref{sec:Conclusion} concludes with future work.

