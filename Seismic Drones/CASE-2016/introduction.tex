\section{Introduction}\label{sec:introduction}

\begin{figure}
\centering
\begin{overpic}[width =\columnwidth]{intro_fig.pdf}\end{overpic}
\caption{\label{fig:OverviewImage}
 Comparing manual and robotic geophone placement. 1a.) Currently, geophones are planted manually [1].   1b.) Traditional methods require extensive cables to connect geophones to the seismic recorders and batteries.  Shown are wire bundles lined up for transportation from the exploration site [2].  1c.) The \emph{Seismic Drone} in this paper is an autonomous unit requiring no external cables.  This paper presents an automated  process for sensor deployment and retrieval.
}
\end{figure}



%Hydrocarbons (coal, oil, natural gas) are estimated to supply more that 66\% of the total energy consumed on earth during the year 2014 by IEA~\cite{IEA16}.  Thus hydrocarbon exploration, the search for hydrocarbons (oil and natural gas deposits) below the earth's surface or sea bed is essential to sustain life on this planet. Millions of dollars are pumped into exploration since these hydrocarbons are major sources of energy, to avoid hazards (maintain safety) as they are inflammable and are an essential part of the socio-economic development.
 
%Traditional exploration involves planting geophones (sensors) into the soil and detecting seismic disturbances caused from a \emph{Veibroseis} trucks (trucks with a heavy metal plate) or dynamite act as the source of vibration, as these vibrations propagate on the surface they are detected by the geophones and the data is stored. The data obtained describes the intensity of the pressure wave generated by the source over a time period and is received by the geophones (sensor). This data is highly useful in analyzing the underground rock structure and inferring the presence of hydrocarbons. Hence instead of randomly searching for hydrocarbons the exploration is carried out by using state of the art techniques, equipment and skilled labor over a large area with potential hydrocarbon reserves. An array of sensors are placed in different patterns while the test is performed, these geophones have a spike and are pushed into to ground to aid the sensing process. Coupling between the sensor and ground is at most important while testing. Since the sensor is coupled with the ground, when the source generates pressure waves the ground oscillates these oscillations are sensed by the sensor and the data is transmitted to the seismic recorder and stored. 

%The current state of the art technology used to perform exploration is bulky and has long cable connections connecting an array of geophones to the seismic recorder. This requires a lot of manual labor, transportation resources, time and energy. These explorations are carried out on thousands of square kilometers of area multiple times. There are emerging technologies that can improve the situation, there are autonomous sensor systems which have spikes and have to be pushed into the ground but instead of an array of sensors which are connected in series to the seismic recorder in general using bulky cable wires, the autonomous node is a single unit comprising the sensor and seismic recorder and battery. It can be deployed at a location for days and collect data, but this data can be viewed only after the experiments are over this is the same case with the extensively used cabled system. This is a drawback since if the data collected was faulty (bad coupling of geophone, the node can be stolen) we would not know until the experiment is over. A recent breakthrough is to wireless sensor nodes and real time data acquisition systems. This improves the system tremendously and using wireless transmission (Radio Frequency) we can cut short the transportation and use of bulky cables. It helps in analyzing complex terrains with mountains, rivers which could not be achieved with the cable system. Real time data acquisition is useful in detecting faults and instant analysis is useful to plan future operations. This system has overcome most of the drawbacks from the cabled systems but this still requires manual labor for setting up their hardware and removing it. The hardware has to be moved periodically during the exploration process to cover the complete field area which again requires manual work. 
 
%The solution is simple, we need to automate the complete process. Quadcopters (flying robots) are extensively used in pick and drop tasks, these robots are gaining incredible popularity amongst the research fraternity and as a commercial product for recreation and delivery tasks. Quadcopters can be the solution the missing piece in the puzzle to automate the process of seismic exploration. Instead of having humans deploy the sensor nodes quadcopters can deploy, retrieve and recharge these sensor nodes by magnetic induction using on board sensors sensors (Camera, lasers, IR) and GPS information. A swarm of robots can perform the task efficiently thereby saving time, resources and decrease the possibility of errors. The major task is to ensure coupling with the surface, which is essential in obtaining quality data.~\cite{MVEwaWSN05},~\cite{CtMiSD08},~\cite{DSSMaA14} 

Hydrocarbons (coal, oil, natural gas) are estimated to
supply more that 66\% of the total energy consumed on earth
during the year 2014 by IEA ~\cite{MVEwaWSN05}.
 Millions of dollars are pumped into exploration since these hydrocarbons are major sources of energy, it is essential for sustaining life and socio-economic developments. Avoiding hazards and maintaining safety is necessary as they are highly-inflammable and human life is at stake thus essentially requiring state of the art equipment to prevent disasters.

Traditional exploration involves planting geophones (sensors)
into the soil and detecting seismic disturbances caused
from Veibroseis trucks or dynamites which act as a source of
vibration. As these vibrations propagate on the surface they
are detected by the geophones and the data is stored. The
data obtained describes the intensity of the pressure wave
generated by the source over a period of time. This data
is critical and is used in analyzing the underground rock
structure and inferring the presence of hydrocarbons. Instead of randomly searching for hydrocarbons, explorations are carried out using elaborate technical procedures, equipment and skilled labor over a large area there by increasing the possibility of discovering hydrocarbon-reserves in an optimal fashion. 
Even though traditional exploration methods are extensively used they are not in par with the current advancements. The use of cables to connect the microprocessor and the sensors leads to drawbacks like increase in overall cost, inaccessibility in certain terrains. The exploration process involves deployment and redeployment of sensors repeatedly manually. With current advancements in automation, automating the process would reduce the expenditure and increase precision.
Drones or unmanned aerial vehicles (UAVs) are flying
platforms with propulsion, positioning, and independent self control.
As drone technology improves and regulations are
adopted, there are major opportunities for their use in scientific
measurement, engineering studies, and education. In particular,
measuring mechanical vibrations is a key component of many
fields, including earthquake monitoring, geotechnical engineering,
and seismic surveying. Seismic imaging is one of the
major techniques (and industries in Texas) for subsurface exploration
and involves generating a vibration which propagates
into the ground, echoes, and is then recorded using motion
sensors. There are numerous sites of resource or rescue interest
that may be difficult or hazardous to access. In addition, there
might be many places to survey, which require a great deal
of hand labor. Thus, there is a substantial need for unmanned
sensors that can be deployed by air and potentially in large
numbers. We have built working prototypes of an Autonomous
Flying Vibration Sensor that can fly to a site, land, then
listen for echoes and vibrations, transmit the information, and
subsequently return to its home base.
The goal of this paper is to design, build, and demonstrate
the use of motion sensing drones for seismic surveys, earthquake monitoring, and remote material testing. 

Section~\ref{sec:RelatedWork}  gives a
description of current state of the technology available in the
industry and why Seismic Drone is better. Followed by the
(??) that describes the hardware experiments performed. (??)
discusses the performance of the above system with traditional methods and followed by (??) Conclusion .

