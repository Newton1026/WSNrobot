\section{Introduction}\label{sec:Introduction}
Seismic surveying is a geophysical technique involving sensor data collection and signal processing. 
It aims at identifying and retrieving hydrocarbons like coal, petrol, natural gas. 
Traditional seismic surveying involves manual laborers placing geophone sensors at specific locations connected by cables. 
Cables are bulky and the amount required is directly proportional to the area surveyed. 
On average hundreds of square kilometers  must be surveyed, requiring many kilometers of cabling. 
Remote locations often require seismic surveying, with concomitant problems of inaccessibility, harsh  conditions, and  transportation of bulky cables and sensors.  
These factors increases the cost. 

  Nodal sensors are a relatively new development to the seismic sensing.
  Nodal sensors are autonomous units that do not require bulky cabling. 
  They have an internal seismic recorder, a micro-controller that records seismic readings from a high-precision accelerometer. 
  Because this technology does not require cabling, the overall cost is reduced. 
  Currently nodal sensors are becoming popular in the USA due to reduced costs in seismic sensing.
  However, these sensors are still planted and recovered by hand.  
  This paper introduces an automated technology for planting and recovering wireless sensors. 
  The technology presented may have wide applicability for quickly deploying sensor assets for geoscience, earthquake monitoring, defense, and  wildlife monitoring. \todo{add citations for each}
  

