\section{Introduction}\label{sec:Introduction}
Seismic surveying is a geophysical technique involving sensor data collection and signal processing. 
%It aides in identifying hydrocarbon reservoirs such as petrol and natural gas. 
 A major application of the method is in the search for subsurface resources. 
Traditional seismic surveying involves manual laborers repeatedly placing geophone sensors at specific locations connected by cables. 
Cables are bulky and the length required is proportional to the area surveyed. 
Surveys routinely cover hundreds of square kilometers, requiring kilometers of cabling. 
Seismic surveying in remote locations has concomitant problems of inaccessibility, harsh  conditions, and  transportation of bulky cables and sensors.  
These factors increase the cost. 

  Nodal sensors, a relatively new development to seismic sensing, are autonomous units that do not require bulky cabling. 
  They have an internal \emph{seismic recorder}, a micro-controller that records seismic readings from a high-precision accelerometer. 
  Because this technology does not require cabling, downtime and overall cost can be reduced. 
  However, these sensors are still planted and recovered by hand.  

\begin{figure}
\centering
\begin{overpic}[width=\columnwidth]{intro.pdf}\end{overpic}
\caption{\label{fig:Hetero_overall}
The heterogeneous sensor system presented in this paper: wireless SeismicDarts and a SeismicSpider, both designed for UAV deployment. 
}
\end{figure}

We propose a heterogeneous robotic system for obtaining seismic data, shown in Fig.~\ref{fig:Hetero_overall}. The system consists of two sensors, the SeismicDart and  the SeismicSpider.  
The SeismicDart is a dart-shaped wireless sensor that is planted in the ground when dropped  from a UAV. 
The SeismicSpider is a mobile hexapod with three legs replaced by geophones.
This system is designed to automate sensor deployment, minimizing cost and time while maximizing accuracy, repeatability, and efficiency.
  The technology presented may have wide applicability where quickly deploying sensor assets is essential, including geoscience research~\cite{werner2006deploying}, 
  earthquake monitoring~\cite{dominici2012micro}, defense operations~\cite{wu2007efficient}, and wildlife monitoring~\cite{dyo2010evolution,mainwaring2002wireless}. 
  
  
  
  
