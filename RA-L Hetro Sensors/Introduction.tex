\section{Introduction}\label{sec:Introduction}
Seismic surveying is a geophysical technique involving sensor data collection and signal processing. 
It aims at identifying and retrieving hydrocarbons like coal, petrol, natural gas. 
Traditional seismic surveying involves manual laborers placing geophone sensors at specific locations connected by cables. 
Cables are bulky and the amount required is directly proportional to the area surveyed. 
On average hundreds of square kilometers  must be surveyed, requiring kilometers of cabling. 
Remote locations often require seismic surveying, with concomitant problems of inaccessibility, harsh  conditions, and  transportation of bulky cables and sensors.  
These factors increases the cost. 

  Nodal sensors are a relatively new development to the seismic sensing.
  Nodal sensors are autonomous units that do not require bulky cabling. 
  They have an internal seismic recorder, a micro-controller that records seismic readings from a high-precision accelerometer. 
  Because this technology does not require cabling, the overall cost is reduced. 
  Currently nodal sensors are becoming popular in the USA due to reduced costs in seismic sensing.
  However, these sensors are still planted and recovered by hand.  
  This paper introduces an automated technology for planting and recovering wireless sensors. 
  The technology presented may have wide applicability for quickly deploying sensor assets for geo-science, earthquake monitoring, defense, and  wildlife monitoring. \todo{add citations for each}

We propose a heterogeneous sensor system for obtaining seismic data. The system consists of two sensors 1.) Seismic spider 2.) Smart dart. The seismic spider is a mobile robot(hexapod)with three of it's legs being replaced by geophones. The smart dart is a dart like sensor that is dropped from the deployment unit. The goal is to automate the process of sensor deployment and thus minimizing factors like cost, time and maximizing accuracy, repeatably and efficiency.
\begin{figure}
\centering
\begin{overpic}[width=\columnwidth]{intro.pdf}\end{overpic}
\caption{\label{fig:Hetero_overall}
Heterogeneous sensor system(smart dart and seismic spider) being carried by the deployment unit.
}
\end{figure}
