 \section{Conclusion and Future Work}\label{sec:Conclusion}
This paper presented an autonomous technique for geophone placement, recording, and retrieval. The system enables automating a job that currently requires large teams of manual laborers. Three components were introduced, SmartDarts, a mobile SeismicSpider, and a deployment unit.
Field and laboratory hardware experiments demonstrated the efficacy of the seismic drone compared to traditional techniques. 
The SmartDart's output were comparable to well-planted geophones, suggesting the feasibility of the proposed system. 
For hard surfaces where the SmartDart could not penetrate, an autonomous alternative was presented, the SeismicSpider.  
The SeismicSpider is mobile, can actively adjust its sensors to ensure ground contact and vertical placement, and can be deployed and retrieved by drone.

Autonomous deployment was conducted using GPS, proving human involvement could be drastically minimized by adopting the proposed technique. 
Hardware experiments compared the autonomous system to manual planting and ballistic deployment.
Simulation studies show time and cost savings over traditional manual techniques.

Future systems should be weatherized and be designed solely for seismic exploration purposes to increase robustness,  range, and speed.
